%%%%%%%%%%%%%%%%%%%%
%%% ALL PACKAGES %%%
%%% ============ %%%

%%%%%%%%%%%%%%%%%%%%%%%%%% GENERAL PACKAGES %%%%%%%%%%%%%%%%%%%%%%%%%
\usepackage[utf8]{inputenc}                                         %   for accepting different input encodings [STANDARD PACKAGE] -- https://www.ctan.org/pkg/inputenc
\usepackage[T1]{fontenc}                                            %   for selecting font encodings [STANDARD PACKAGE] -- https://www.ctan.org/pkg/fontenc
\usepackage[english]{babel}                                         %   for english and german language -- https://www.ctan.org/pkg/babel
\usepackage{datetime}                                               %   for date and time   -- https://www.ctan.org/pkg/datetime
\usepackage{lmodern}                                                %   for text font as Latin Modern -- https://www.namsu.de/Extra/pakete/Lmodern.html
\usepackage[left=2cm,right=2cm,top=2cm,bottom=3cm]{geometry}        %   for page/geometry layout -- https://www.ctan.org/pkg/geometry
\usepackage{fancyhdr}                                               %   for fancy template  -- https://www.ctan.org/pkg/fancyhdr
\usepackage[Glenn]{fncychap}                                        %   for fancy chapter style -- https://ctan.org/pkg/fncychap
\usepackage{import}                                                 %   for import of other files, i.e. a .tex file with pgfplots  -- https://www.ctan.org/pkg/import
\usepackage[backend=bibtex,style=nature,sorting=none]{biblatex}  %   for using bibliography, references and citations -- https://ctan.org/pkg/biblatex
\usepackage[babel=true]{csquotes}                                   %   for using quotations -- https://ctan.org/pkg/csquotes
\usepackage{epigraph}                                               %   for quotes at the beginning of a chapter -- https://ctan.org/pkg/epigraph
\usepackage{comment}                                                %   for extra comment features  -- https://www.ctan.org/pkg/comment
\usepackage{float}                                                  %   for positioning figures and tables -- https://ctan.org/pkg/float
\usepackage[leftcaption]{sidecap}                                   %   for positioning caption next to figures -- https://ctan.org/pkg/sidecap
\usepackage[section]{placeins}                                      %   for controlling float placement -- https://ctan.org/pkg/placeins
%%%%%%%%%%%%%%%%%%%%%%%%%%%%%%%%%%%%%%%%%%%%%%%%%%%%%%%%%%%%%%%%%%%%%

%%%%%%%%%%%%%%%%% MATH, SCIENCE AND SPECIAL SYMBOL PACKAGES %%%%%%%%%%%%%%%%%%%%%
\usepackage{amsmath,amsfonts,amssymb,amsthm,nccmath,bbm,mathdots,mathrsfs}      %   for mathematics, i.e. bbm for identity matrix
%   -- https://www.ctan.org/pkg/amsmath                                         %
%   -- https://www.ctan.org/pkg/amsfonts                                        %
%   -- https://www.ctan.ebinger.cc/tex-archive/fonts/amsfonts/doc/amssymb.pdf   %
%   -- https://www.ctan.org/pkg/amsthm                                          %
%   -- https://www.ctan.org/pkg/nccmath                                         %
%   -- https://www.ctan.org/pkg/bbm                                             %
%   -- https://www.ctan.org/pkg/mathdots                                        %
%   -- https://www.ctan.org/pkg/mathrsfs                                        %
\usepackage[nointegrals]{wasysym}                                               %   for more symbols, i.e. astronomical symbols -- https://www.ctan.org/pkg/wasysym
\usepackage{physics}                                                            %   for physics, i.e. brakets -- https://www.ctan.org/pkg/physics
\usepackage{siunitx}                                                            %   for using si-units -- https://www.ctan.org/pkg/siunitx
\usepackage{fontawesome}                                                        %   for fontawesome symbols -- https://www.ctan.org/pkg/fontawesome
                                                                                %   for mathematical enhancements in LaTeX (American Mathematical Society), have a look at https://www.ctan.org/pkg/amslatex
                                                                                %   for mathematical and scientific symbols, have a look a at 'The Comprehensive LaTeX Symbol List' -- https://ftp.mpi-inf.mpg.de/pub/tex/mirror/ftp.dante.de/pub/tex/info/symbols/comprehensive/symbols-a4.pdf
%%%%%%%%%%%%%%%%%%%%%%%%%%%%%%%%%%%%%%%%%%%%%%%%%%%%%%%%%%%%%%%%%%%%%%%%%%%%%%%%%

%%% COLORS AND GRAPHICS PACKAGES %%%
\usepackage{xcolor}                %   for using colors -- https://www.ctan.org/pkg/xcolor
\usepackage{empheq}                %   for emphasizing equations -- https://www.ctan.org/pkg/empheq
\usepackage[most]{tcolorbox}       %   for colored boxes - https://www.ctan.org/pkg/tcolorbox
\usepackage{realboxes}             %   for more box options -- https://www.ctan.org/pkg/realboxes
\usepackage{graphicx}              %   for using graphics -- https://www.ctan.org/pkg/graphicx
\usepackage{eso-pic}               %   for picture options -- https://www.ctan.org/pkg/eso-pic
\usepackage{transparent}           %   for transparency in pictures -- https://www.ctan.org/pkg/transparent
\usepackage{standalone}            %   for compiling pictures and graphics in a seperate .tex-file and including it in the main document -- https://www.ctan.org/pkg/standalone
\usepackage{tikz}                  %   for creating beatiful and precise graphics, i.e. picture of spherical coordinates -- https://www.ctan.org/pkg/pgf
\usepackage{pgfplots}              %   for creating plots in two or three dimensions -- https://www.ctan.org/pkg/pgfplots
%%%%%%%%%%%%%%%%%%%%%%%%%%%%%%%%%%%%

%%% NUMERATE, HYPERLINK AND HIGHLIGHTING PACKAGES %%%
\usepackage{caption}                                %   for more caption options in graphics -- https://www.ctan.org/pkg/caption
\usepackage{enumerate}                              %   for numerating, i.e. sections -- https://www.ctan.org/pkg/enumerate
\usepackage{footnotebackref}                        %   for hyperlinks of footnotes -- https://www.ctan.org/pkg/footnotebackref
\usepackage{hyperref}                               %   for highlighting/reference of links, i.e. websites -- https://www.ctan.org/pkg/hyperref
\usepackage{listings}                               %   for typesetting code -- https://www.ctan.org/pkg/listings
\usepackage{circledsteps} %for circled numbers etc.
%\usepackage{minted}                                 %   for highlighting code -- https://www.ctan.org/pkg/minted
%\usepackage{pythonhighlight}                        %   for highlighting python code -- https://www.ctan.org/pkg/pythonhighlight
%%%%%%%%%%%%%%%%%%%%%%%%%%%%%%%%%%%%%%%%%%%%%%%%%%%%%


\usepackage{blindtext}
\newcommand{\bt}{\blindtext}


%%%%%%%%%%%%%%%%%%%%
%%% OWN COMMANDS %%%
%%% ============ %%%

%%% MATH COMMANDS %%%
% -- own commands for math symbols -- %
\newcommand{\N}{\mathbb{N}}           %   for the set of natural numbers
\newcommand{\Z}{\mathbb{Z}}           %   for the set of integers
\newcommand{\Q}{\mathbb{Q}}           %   for the set of rational numbers
\newcommand{\R}{\mathbb{R}}           %   for the set of real numbers
\newcommand{\C}{\mathbb{C}}           %   for the set of complex numbers
\newcommand{\D}{\mathrm{d}}           %   for mathroman d, i.e. differential d
\newcommand{\E}{\mathrm{e}}           %   for mathroman e, i.e. Euler's constant
\newcommand{\I}{\mathrm{i}}           %   for mathroman i, i.e. imaginary unit
\newcommand{\1}{\mathbbm{1}}          %   for identity matrix
\newcommand{\bs}{\boldsymbol}         %   for bold math symbols (instead of \textbf{} or \mathbf{}), i.e. for vectors in physcs 
\DeclareMathOperator{\arsinh}{arsinh} %   for arsinh 
\DeclareMathOperator{\diag}{diag}     %   for writing a diagonal matrix, i.e. sign convetion of Minkowski metric tensor \eta_{\mu \nu} = \diag(-,+,+,+)
\newcommand{\latex}{\LaTeX\xspace}    %   for the LaTeX symbol
\newcommand\mathbbf[2][.2]{           %   for even bolder mathsymbols
  \def\thickness{#1}
  \ThisStyle{\outline{$\mathbf{\SavedStyle#2}$}}
}
\newcommand{\CR}{\hat a^\dagger}
\newcommand{\AN}{\hat a}
\newcommand{\defeq}{\vcentcolon=}
\newcommand{\eqdef}{=\vcentcolon}
\newcommand{\ham}{\hat{\mathcal{H}}}
\newcommand{\Sum}{\sum\limits}
\newcommand{\NO}{\vcentcolon}
\newcommand{\eqsec}{\overset{\CircledTop{2}}{=}}



% -- own command for theoremstyle, see amsthm -- %
\theoremstyle{plain}                             %
\newtheorem{definition}{Definition}              %
%%%%%%%%%%%%%%%%%%%%%


%%% E-MAIL COMMAND %%%
% -- own command to insert e-mail -- %
\newcommand{\email}[2]{\href{mailto:#1}{#2}}
%%%%%%%%%%%%%%%%%%%%%

%%% FOOTNOTE COMMAND %%%
% -- own command for foonotes -- %
%\renewcommand*{\thefootnote}{\fnsymbol{footnote}}
\renewcommand*{\thefootnote}{[\arabic{footnote}]}
%%%%%%%%%%%%%%%%%%%%%%%%

%%% COMMANDS FOR TITLEPAGE %%%
\newcommand*{\getAuthor}{Jan-Philipp Anton Konrad Christ}
\newcommand*{\getSupervisorOne}{Prof. Dr. Fabian Bohrdt, geb. Grusdt}
\newcommand*{\getSupervisorTwo}{...}
% \newcommand*{\getExamDate}{Date of final exam}
% -- english -- %
\newcommand{\getTitleEN}{Title of My Thesis}
\newcommand{\getSubtitleEN}{Maybe with some Subtitle}
\newcommand*{\getPrintLocationEN}{Munich}
\newcommand*{\getPrintYearEN}{\the\year}
\newcommand*{\getPlaceOfBirthEN}{Landau/ Pfalz}
\newcommand*{\getSubmissionDateEN}{22.06.2023}
\newcommand*{\langEN}{en-US}
% -- german -- %
\newcommand{\getTitleDE}{Titel meiner Arbeit}
\newcommand{\getSubtitleDE}{Vielleicht mit Untertitel}
\newcommand*{\getPrintLocationDE}{München} 
\newcommand*{\getPrintYearDE}{\the\year}
\newcommand*{\getPlaceOfBirthDE}{Landau/ Pfalz}
\newcommand*{\getSubmissionDateDE}{22.06.2023}
%\newcommand*{\langDE}{de}
%%%%%%%%%%%%%%%%%%%%%%%%%%%%%%%

%%% COMMAND FOR BACKGROUND PICTURE %%%
% source: https://tex.stackexchange.com/questions/86500/includegraphics-set-image-opacity
% \newcommand\BackgroundPic{%
% \put(200,200){%
%             \parbox[b][\paperheight]{\paperwidth}{
%             \vfill
%             \centering
%             {
%             \transparent{0.2}
%             \includegraphics[width=\paperwidth, height=\paperheight, keepaspectratio]{figures/lmu-siegel.png}
%             }
%             \vfill
%         }
%     }
% }
%%%%%%%%%%%%%%%%%%%%%%%%%%%%%%%%%%%%%%

%%% COMMAND FOR PRETTY CODE HIGHLIGHTING %%%
% source: https://tex.stackexchange.com/questions/140166/making-inline-code-printing-pretty?noredirect=1&lq=1
% \newcommand\code[3][]{
%     \tikz[baseline=(s.base)]{
%         \node(s)[
%             rounded corners,
%             fill=blue!5,        % background color
%             draw=gray,          % border of box
%             %text=gray!50!black, % text color
%             inner xsep =3pt,    % horizontal space between text and border
%             inner ysep =0pt,    % vertical space between text and border
%             text height=2ex,    % height of box
%             text depth =1ex,    % depth of box
%             #1                  % other options
%         ]{\mintinline{#2}{#3}};
%     }
% }
%%%%%%%%%%%%%%%%%%%%%%%%%%%%%%%%%%%%%%%%%%%

%%% COMMAND FOR EMPTY PAGE %%%
% source: https://tex.stackexchange.com/questions/34934/add-a-new-empty-page
%\newcommand*\NewPage{\newpage\null\thispagestyle{empty}\newpage}
%%%%%%%%%%%%%%%%%%%%%%%%%%%%%%


%%%%%%%%%%%%%%%%%%%%%%
%%% PACKAGE SETUPS %%%
%%% ============== %%% 

%%% BIBLATEX SETUP %%%
% source: https://tex.stackexchange.com/questions/534565/apply-empty-style-to-the-entire-bibliography
\renewcommand{\bibsetup}{\thispagestyle{empty}}
\makeatletter
\patchcmd{\blx@endenv@bibliography}{\endlist}{\endlist\clearpage}{}{}
\makeatother
%%%%%%%%%%%%%%%%%%%%%%

%%% CLASS SETUP %%%
\setcounter{secnumdepth}{3}
\setcounter{tocdepth}{3}
%%%%%%%%%%%%%%%%%%%

%%% CSQUOTES SETUP %%%
% source: https://tex.stackexchange.com/questions/329334/quote-inline-with-italics
% \DeclareQuoteStyle[american]{english}
% {\itshape\textquotedblleft}
% [\textquotedblleft]
% {\textquotedblright}
% [0.05em]
% {\textquoteleft}
% {\textquoteright}
%%%%%%%%%%%%%%%%%%%%%%

%%% FANCYHDR SETUP %%% 
\pagestyle{fancy}
\fancyhf{}
\fancyhead[RE]{\normalfont\bfseries\sffamily\leftmark}
\fancyhead[LO]{\normalfont\bfseries\sffamily\rightmark}
\fancyhead[RO,LE]{\thepage}

\fancypagestyle{plain}{%
  %\fancyhf{}
  %\fancyhead[RO,LE]{\thepage}
}

\ChTitleVar{\bfseries\Large\rmfamily}
\ChNameVar{\Large\sffamily}

\makeatletter
\renewcommand\chaptermark[1]{%
  \markboth{%
    \ifnum \c@secnumdepth >\m@ne
      \if@mainmatter
        %\chaptername
      \fi
    \fi
    #1%
  }{}
}

\renewcommand\sectionmark[1]{\markright{\thesection\enspace #1}}
%\renewcommand\subsectionmark[1]{\markright{\thesubsection\enspace #1}}
\makeatother          
%%%%%%%%%%%%%%%%%%%%%%

%%% FONT SETUP %%%
%\renewcommand{\familydefault}{\sfdefault}
%%%%%%%%%%%%%%%%%%

%%% HYPER SETUP %%%
\hypersetup{   
    pdfpagemode = {UseNone},
    pdftitle = {\getTitleEN},
    pdfauthor = {\getAuthor},
    pdflang = {\langEN},
    colorlinks = true,     
    linkcolor = black,      
    citecolor = black,
    filecolor = black,      
    urlcolor = black       
}                        
%%%%%%%%%%%%%%%%%%%

%%% LISTINGS SETUP %%%
 % \definecolor{mygreen}{rgb}{0,0.6,0}
 % \definecolor{mygrey}{rgb}{0.8,0.8,0.8}
 % \definecolor{mymauve}{rgb}{0.58,0,0.82}

 % \lstset{ %
 %   backgroundcolor=\color{mygrey},       % choose the background color
 %   basicstyle=\ttfamily,                 % size of fonts used for the code
 %   breaklines=true,                      % automatic line breaking only at whitespace
 %   captionpos=b,                         % sets the caption-position to bottom
 %   commentstyle=\color{mygreen},         % comment style
 %   escapeinside={\%*}{*)},               % if you want to add LaTeX within your code
 %   keywordstyle=\color{blue},            % keyword style
 %   numbers=left,                         % alignment of numbers
 %   numberstyle={\small \color{black}},    % number style
 %   numbersep=9pt,                        % this defines how far the numbers are from the text
 %   stepnumber=1,
 %   stringstyle=\color{mymauve}           % string literal style
 % }
%%%%%%%%%%%%%%%%%%%%%

%%% SI SETUPS %%%
\sisetup{locale=US}                     
\sisetup{per-mode=symbol-or-fraction} 
\DeclareSIUnit \au {au}
\DeclareSIUnit \ly {ly}
\DeclareSIUnit \parsec {pc}
\DeclareSIUnit \yr {yr}
%%%%%%%%%%%%%%%%%

%%% SIDECAP SETUP %%%
% \sidecaptionvpos{figure}{t}
%%%%%%%%%%%%%%%%%%%%%

%%% TCOLORBOX SETUP %%%
% \definecolor{mygray}{rgb}{0.8,0.8,0.8}
% \tcbset{
%         on line,
%         boxsep = 2pt,
%         left = 0pt,
%         right = 0pt,
%         top = 0pt,
%         bottom = 0pt,
%         colframe = white,
%         colback = mygray
% }
%%%%%%%%%%%%%%%%%%%%%%%%

%%% TIKZ AND PGFPLOTS SETUP %%%
% \usetikzlibrary{intersections}
% \usetikzlibrary{decorations.pathreplacing, decorations.markings}
% \usetikzlibrary{tikzmark}
% \pgfplotsset{every axis/.style={scale only axis}, compat=newest}
%%%%%%%%%%%%%%%%%%%%%%%%%%%%%%%

%%% TITLESEC SETUP %%%
% source: https://tex.stackexchange.com/questions/111643/decrease-space-before-and-after-chapter-in-fncychap
\makeatletter
\patchcmd{\@makechapterhead}{\vspace*{50\p@}}{\vspace*{-20\p@}}{}{}
\patchcmd{\@makeschapterhead}{\vspace*{50\p@}}{\vspace*{-20\p@}}{}{}
\patchcmd{\DOTI}{\vskip 80\p@}{\vskip 40\p@}{}{}
\patchcmd{\DOTIS}{\vskip 40\p@}{\vskip 0\p@}{}{}
\makeatother
% \makeatletter
% \titleformat{\chapter}[frame]
%   {\normalfont}{\filright\enspace \@chapapp~\thechapter\enspace}
%   {8pt}{\LARGE\bfseries\filcenter}
% \titlespacing*{\chapter}
%   {0pt}{0pt}{20pt}
% \makeatother
%%%%%%%%%%%%%%%%%%%%%%

%%% TOC SETUP %%%
% source: https://tex.stackexchange.com/questions/5787/table-of-contents-with-page-style-empty
\AtBeginDocument{\addtocontents{toc}{\protect\thispagestyle{empty}}} 
%%%%%%%%%%%%%%%%%

% Change Chapter xy to sth else
\makeatletter
\renewcommand{\@chapapp}{Section}
\makeatother

%Command for Julia Logo
\newcommand{\julialogo}{\raisebox{-0.5\dp\strutbox}{\includegraphics[height=1em]{figures/julialogo.pdf}}}

