\chapter{Deriving the Flow Equations}\label{Determining the Flow Equations}
\section{Purely Quadratic Case}
\subsection{Applying the Formalism}
In the purely quadratic case where the coefficients do not depend on the occupation numbers and where a static impurity is considered, exact flow equations \ref{feq_won_I}-\ref{feq_won_IV} can be derived. In particular, the flow Hamiltonian $\ham(\lambda)$ is of the same quadratic form as the original Hamiltonian
\begin{equation}\label{def_purely_quadratic_hamiltonian}
\ham \defeq \ham_0+\ham_{\mathrm{int}} \defeq \Sum_k \omega_k \CR_k\AN_k+\Sum_{q\neq q^\prime}V_{q,q^\prime}\CR_q\AN_{q^\prime}+\Sum_{p,p^\prime}\left(W_{p,p^\prime}\CR_p\CR_{p^\prime}+\mathrm{h.c.}\right)
\end{equation}
 and no truncation scheme as discussed in Section \ref{Truncation Schemes} has to be employed. For details of the calculations consult Section \ref{Deriving the Flow Equations in the Case of No n-Dependence}.\par 
We followed precisely the recipe introduced in Section \ref{General Mechanism}: First the canonical generator \ref{FEQ_eta_def} is calculated (see equation \ref{eta_wo_n_final}). Then the derivative of the flow Hamiltonian is calculated via the commutator of the generator and the Hamiltonian in equation \ref{FEQ_ODE_def}. The relevant terms \ref{wo_n_I}-\ref{wo_n_IV} are evaluated separately by only making use of the bosonic commutation relations. The flow equations then follow by a simple comparison of coefficients from the quadratic Hamiltonian (plus a constant energy $\epsilon$) ansatz for the flow Hamiltonian.\par
Note that the resulting flow equations \ref{feq_won_I}-\ref{feq_won_IV} suggest that they are exact in the sense that if the flow is completely traversed, the flow Hamiltonian will be exactly diagonal because in first order
\begin{subequations}
\begin{align}
V_{q,q^\prime}&\sim \exp\left(-(\omega_q-\omega_{q^\prime})^2\right)\xlongrightarrow{\lambda\rightarrow\infty}0\\
W_{p,p^\prime}&\sim \exp\left(-(\omega_p+\omega_{p^\prime})^2\right)\xlongrightarrow{\lambda\rightarrow\infty}0,
\end{align}
\end{subequations}
assuming that there are no degeneracies. If there were, it would mean that  $(\omega_q-\omega_{q^\prime})^2=0$ for some pair $q,q^\prime$ and that the corresponding matrix element $V_{q,q^\prime}$ is not exponentially suppressed in first order of the flow equations. Alternatively, near-degeneracies may render the convergence so slow that the corresponding matrix elements decay so slowly that the flow would have to be traversed an unreasonable or impractical distance to achieve sufficient suppression.  Fortunately, the flow equations \emph{can} still be successful in some cases \cite{PhysRevD.49.4214} because the second order terms coupling the ODEs for the different matrix elements are non-trivial and might, depending on the initial condition, even sufficiently suppress degenerate matrix elements.
It follows that applying the equations to a concrete problem provides the best test of their performance and convergence properties.\\
Also, checking the condition \ref{check_trace} is indeed a strong indicator for good convergence properties but does not necessarily imply that all elements in $\ham_{\mathrm{int}}$ converge to 0 when degeneracies are present, which is why we do not explicitly evaluate of the conditions \ref{useless_check} which in turn imply \ref{check_trace}.
\subsection{Application to the 1D Bose Polaron Model}
In the heavy impurity limit $M\rightarrow \infty$, the dependence of the occupation numbers in the LLP-Hamiltonian \ref{ham_LLP} vanishes and we get an Hamiltonian which differs from the purely quadratic form we discussed in the last section only by linear terms. Before we discuss the significance of these linear term we will address the fact that the integrals in there have to discretized for numerical treatment.\\
To this end, we will restrict ourselves to a discrete number of modes $k$ where $0<\Lambda_{IR}\leq |k|\leq\Lambda_{UV}<\infty$ . $\Lambda_{IR}$ denotes the infrared and $\Lambda_{UV}$ denotes the ultraviolet cut-off. We will work with values $\Lambda_{IR}\xi= 10^{-1}$ and $\Lambda_{UV}\xi= 10^1$. \\
Considering a larger number of $k$ values is generally better, but involves significant computational cost. That is why the spacing $\Delta k=\frac{2\pi}{L}$ (where $L$ is a constant which describes the size of the system) between two adjacent $k$ values will be not be chosen too small. Typical values are of order $\Delta k\sim 10^{-1}\xi$.  \\
These values are small enough that we are allowed to approximate integrals by sums
\begin{equation}
\int\mathrm d k\rightarrow \Delta k\Sum_k
\end{equation}
over a discrete and finite grid of size $N\in\N$.
The commutation relations of the creation and annihilation operators in $\ham_{\mathrm{LLP}}$ are $\left[\AN_k,\CR_{k^\prime}\right]=\delta(k-k^\prime)$. Our new discrete operators will obey the same commutation relation with a Kronecker-delta instead of the Dirac-delta.
The transition from the continuous to the discrete case is done by coarsening the annihilation and creation operators
\begin{equation}
\AN_k^{\left(\dagger\right)}\rightarrow \frac1{\sqrt{\Delta k}}\AN_k^{\left(\dagger\right)}
\end{equation}
(see \cite[eq. (7)]{PracticalTraining})
and the LLP-Hamiltonian becomes:
\begin{align}\label{ham_LLP_discrete}
\ham_{\mathrm{LLP}}^{\mathrm{discr.}} &=g_{IB}n_0+\Sum_k \omega_{ k}\CR_{ k}\AN_{ k}+\sqrt{\frac{n_0\Delta k}{2\pi}}g_{IB}\Sum_k W_k \left(\AN_{ k}+\CR_{- k}\right) \\ &+ \frac{g_{IB}\Delta k}{2\pi}\Sum_{k,k^\prime} \left(c_k\CR_k-s_k\AN_{-k}\right)\left(c_{k^\prime}\AN_{k^\prime}-s_{k^\prime}\CR_{-k^\prime}\right)\nonumber
\end{align}
It would be conceivable to extend our flow equations to allow for linear terms in our Hamiltonian. In this case, it would again be possible to obtain a closed set of flux equations. \\
However, this can be avoided because the linear terms  $W_k \left(\AN_{ k}+\CR_{- k}\right)$ can be eliminated by applying the displacement operator 
\begin{equation}
\hat D(\underline\alpha)\defeq\exp\left(\sum_k\alpha_k\CR_k-\mathrm{h.c.}\right)=\exp\left(-\underline\alpha^\dagger{\underline{\underline\Omega}}\ \underline{\hat a}\right)
\end{equation}
to the discrete LLP-Hamiltonian \cite{PracticalTraining}. Here we introduced the symplectic $2N\times2N$ matrix
\begin{equation}\label{symplectic_matrix}
{\underline{\underline\Omega}}=\begin{pmatrix}\underline{\underline E}_N & 0\\ 0 & -\underline{\underline E}_N\end{pmatrix}
\end{equation}
and the notation
\begin{equation}\label{cr_an_vector}
\underline{\hat a} = \left(\AN_{k_1},...,\AN_{k_N},\CR_{k_1},...,\CR_{k_N}\right)^T
\end{equation}
for vectors of creation and annihilation operators as well as
\begin{equation}\label{c-number_vector}
\underline\alpha = \left(\alpha_{k_1},...,\alpha_{k_N},\alpha_{k_1}^*,...,\alpha_{k_N}^*\right)^T\in\C^{2N}
\end{equation}
for vectors of c-numbers. \par %In this context $N$ is equal to the number of modes on our discrete grid.\par
The displacement operator shifts creation and annihilation operators by a given c-number:
\begin{subequations}
\begin{align}
\hat D^\dagger(\underline \alpha)\AN_{k_i}\hat D(\underline\alpha)&=\AN_{k_i}+\alpha_{k_i}\\
\hat D^\dagger(\underline \alpha)\CR_{k_i}\hat D(\underline\alpha)&=\CR_{k_i}+\alpha_{k_i}^*
\end{align}
\end{subequations}
This can be proved readily with the help of the Baker-Campbell-Hausdorff formula.
Simliarly, one can convince oneself that
\begin{subequations}
\begin{align}
\hat D(\underline \alpha)\AN_{k_i}\hat D^\dagger(\underline\alpha)&=\AN_{k_i}-\alpha_{k_i}\\
\hat D(\underline \alpha)\CR_{k_i}\hat D^\dagger(\underline\alpha)&=\CR_{k_i}-\alpha_{k_i}^*.
\end{align}
\end{subequations}
From this it follows immediately that the displacement operator is unitary, and therefore applying it to the discrete LLP-Hamiltonian does not change its spectrum.
We obtain:
\begin{align}
\hat D^\dagger(\underline \alpha)\ham_{\mathrm{LLP}}^{\mathrm{discr.}}\hat D(\underline\alpha) &=g_{IB}n_0+\Sum_k \omega_{ k}(\CR_{ k}+\alpha_k^*)(\AN_{ k}+\alpha_k)\nonumber\\
&+\sqrt{\frac{n_0\Delta k}{2\pi}}g_{IB}\Sum_k W_k \left(\AN_{ k}+\CR_{- k}+\alpha_k+\alpha_{-k}^*\right) \nonumber\\ 
&+ \frac{g_{IB}\Delta k}{2\pi}\Sum_{k,k^\prime} \left(c_k(\CR_k+\alpha_k^*)-s_k(\AN_{-k}+\alpha_{-k})\right)\left(c_{k^\prime}(\AN_{k^\prime}+\alpha_{k^\prime})-s_{k^\prime}(\CR_{-k^\prime}+\alpha_{-k^\prime}^*)\right)
\end{align}
After using of the symmetry $W_k=W_{-k}$ and the associated symmetries for $c_k$ and $s_k$ and reordering the terms, the condition that the displacement transformation turns our discrete LLP-Hamiltonian into a purely quadratic Hamiltonian reads:
\begin{equation}\label{quadratic_condition_displacement}
\forall k:\ 0\overset{!}{=}\omega_k\alpha_k^*+\tilde W_k^{(0)}+\Sum_{k^\prime}V_{k^\prime,k}^{(0)}\alpha_{k^\prime}^*+\Sum_{k^\prime}\alpha_{k^\prime}\left(W_{k,k^\prime}^{(0)}+W_{k^\prime,k}^{(0)}\right)
\end{equation}
Here we defined
\begin{subequations}
\begin{align}
\tilde W_k^{(0)}&\defeq \frac{g_{IB}}{2\pi}\sqrt{n_0\Delta k} W_k\\
V_{k,k^\prime}^{(0)}&\defeq\frac{g_{IB}}{2\pi}\Delta k(c_k c_{k^\prime}+s_{k}s_{k^\prime})\\
W_{k,k^\prime}^{(0)}&\defeq -\frac{g_{IB}}{2\pi}\Delta k s_k c_{k^\prime}
\end{align}
\end{subequations}
adopting the notation in the generic quadratic Hamiltonian \ref{def_purely_quadratic_hamiltonian}.\\
The condition \ref{quadratic_condition_displacement} can be solved very efficiently and inexpensively using existing solvers for linear systems of equations, which is why the approach involving the displacement operator is generally favored to solving a larger set of ODEs to also suppress the linear parts in the flow. The solution $\underline\alpha$ obtained this way can then be substituted into the Hamiltonian which is then of purely quadratic form: \footnote{The relevant source code "displacement\_transformation.jl" for performing the displacement transformation and initializing the parameters can be found in the GitHub Repository \url{https://github.com/SufficientlySmooth/Bachelor-Thesis-Numerics}.}
\begin{equation}\label{Ham_LLP_qudratic}
\ham_{\mathrm{LLP}}^{\mathrm{quadr.}}=\Sum_k (\omega_k +V_{k,k}^{(0)})\CR_k\AN_k+\Sum_{q\neq q^\prime}V_{q,q^\prime}^{(0)}\CR_q\AN_{q^\prime}+\Sum_{p,p^\prime}\left(W_{p,p^\prime}^{(0)}\CR_p\CR_{p^\prime}+\mathrm{h.c.}\right)+g_{IB}n_0+\frac{g_{IB}}{2\pi}\Delta k\Sum_{k}s_k^2
\end{equation}
The flow equations \ref{feq_won_I}-\ref{feq_won_IV} for $\ham_{\mathrm{LLP}}^{\mathrm{quadr.}}$ define a system of $(2N^2+N+1)\in\mathcal O(N^2)$ ODEs and can be solved numerically using preexisting ODE solvers. We will use the \verb!ODEProblem! class from \julialogo's \verb!DifferentialEquations.jl! library in combination with the \verb!Tsit5! integrator, a Runge-Kutta integrator of order 5(4) \cite{TSITOURAS2011770}, for its general robustness and versatility\footnote{For the numerics we again refer to the source code "Solve\_Flow\_Equations.jl" in \url{https://github.com/SufficientlySmooth/Bachelor-Thesis-Numerics}}.\par
Finally, concrete parameters for $c,\xi,\eta,\gamma,n_0$ have to be chosen. We refer to Catani's experimental results and set $\gamma=0.438$, $n_0\xi=1.05$ and choose our units s.t. $c=\xi=1$ \cite{Catani,Grusdt_2017}.
Then the other constants introduced before can be calculated using the expressions introduced in Section \ref{(Beyond) The Fröhlich Hamiltonian}.
\subsection{Benchmark: Exact Diagonalization via Bogoliubov Transformation}
The purely quadratic Hamiltonian \ref{Ham_LLP_qudratic} can be exactly diagonalized by Bogoliubov Transformations \cite{PracticalTraining,1980ZPhyB..38..271H,PhysRevA.98.033610}. Using the short hand notations \ref{cr_an_vector} and \ref{c-number_vector} as well as \ref{symplectic_matrix}, our quadratic Hamiltonian can be written in the form:
\begin{align}
\ham &= E_0 + \Sum_{k,k^\prime}\CR_k A_{k,k^\prime}\AN_{k^\prime}+\frac12\Sum_{k,k^\prime}\left(\CR_k B_{k,k^\prime}\CR_{k^\prime}+\mathrm{h.c.}\right)\\
&=E_0-\frac12\Sum_k A_{k,k}+\frac12\underline\CR\begin{pmatrix}A & B\\ B^* & A^*\end{pmatrix}\underline\AN\\
&=E_0-\frac12\Sum_k A_{k,k}+\frac12\underline{\CR}\ {\underline{\underline{\Omega}}}\ {\underline{\underline{\mathscr{H}}}}\ {\underline{\AN}}
\end{align}
where we introduced
\begin{equation}
{\underline{\underline{\mathscr{H}}}}\defeq\begin{pmatrix}A & B\\ -B^* & -A^*\end{pmatrix}\in\C^{2N\times 2N}
\end{equation}
and
\begin{equation}
A=\left(A_{k,k^\prime}\right)_{k,k^\prime=1,...,N},\ B=\left(B_{k,k^\prime}\right)_{k,k^\prime=1,...,N}.
\end{equation}
Hermiticity of the Hamiltonian requires $A^\dagger=A$ and $B$ can and must always be chosen s.t. $B^T=B$. This is because $[\CR_k,\CR_{k^\prime}]=[\AN_k,\AN_{k^\prime}]=0$, so if we start with a non-symmetric $B$ we can always symmetrize $B\rightarrow \frac12\left(B+B^T\right)$.\\
As shown in \cite{PracticalTraining}, the Bogoliubov Transformation
\begin{equation}\label{Bogoliubov Transformation Def} \underline\AN\mapsto  \underline{\underline{U_B}}\ \underline\AN\end{equation}
defined by the matrix 
\begin{equation}\label{symplectic_U_B}
\underline{\underline{U_B}}\defeq \begin{pmatrix}U^* & -V^*\\ -V & U\end{pmatrix}
\end{equation}
conserves the bosonic commutation relations iff $\underline{\underline{U_B}}$ is a symplectic matrix:
\begin{equation}\underline{\underline{U_B}}\ {\underline{\underline{\Omega}}}\ \underline{\underline{U_B}}^\dagger={\underline{\underline{\Omega}}}.\end{equation}
There exists $\underline{\underline{U_B}}$ s.t.
\begin{equation}\label{diagonalization_bog_crux}
\underline{\underline{U_B}}^\dagger {\underline{\underline{\Omega}}}\ {\underline{\underline{\mathscr{H}}}}\ \underline{\underline{U_B}}=\mathrm{diag}\left(\lambda_1,...,\lambda_N,\lambda_1^*,...,\lambda_N^*\right)
\end{equation}
The values $\{\lambda_j^{(*)}\}_{j=1,...,N}$ can by obtained by solving the eigenvalue problem of ${\underline{\underline{\mathscr{H}}}}$. The eigenvectors then constitute the columns $\underline{\underline{U_B}}$. The ground state energy reads (at least if all eigenvalues are real):
\begin{equation}\label{GS_energy_formula}
E_{GS}=E_0-\frac12\Sum_k A_{k,k}+\Sum_k \lambda_k
\end{equation}
Additionally, we can make use of the fact that the real eigenvalues always occur in pairs $(\lambda_j,-\lambda_j^*)$. To see this, we introduce the operator \begin{equation}\underline{\underline{\vartheta}}:\C^{2N}\rightarrow \C^{2N},\begin{pmatrix} \underline u\\ \underline v\end{pmatrix}\mapsto \begin{pmatrix} \underline v^*\\ \underline u^*\end{pmatrix}\ \mathrm{where}\ \underline u,\underline v\in\C^N.\end{equation}
The relations 
\begin{equation}\label{theta_Omega_relation}
\left\{\underline{\underline{\vartheta}},{\underline{\underline{\Omega}}}\right\}=0
\end{equation}
and \begin{equation} \left[ {\underline{\underline{\Omega}}}\ {\underline{\underline{\mathscr{H}}}},\underline{\underline{\vartheta}}\right]=0\end{equation}
imply that if $\underline x$ is an eigenvector of ${\underline{\underline{\Omega}}}\ {\underline{\underline{\mathscr{H}}}}$ corresponding to the eigenvalue $\lambda$, then $\underline{\underline{\vartheta}}\ \underline x$ is an eigenvector corresponding to the eigenvalue $-\lambda^*$.
The $ \lambda_j$ in equation \ref{diagonalization_bog_crux} are characterized by the fact that their associated eigenvectors $\underline w_j$ have positive matrix elements:
\begin{equation}
\underline w_j^\dagger{\underline{\underline{\Omega}}}\ \underline w_j>0
\end{equation}
The corresponding eigenvector to $-\lambda_j^*$ always has a negative matrix element. This follows immediately from the relation \ref{theta_Omega_relation} and the fact that obviously $\underline{\underline{\vartheta}}^2=\underline{\underline{E}}_{2N}$.\\
It is necessary to classify the eigenvalues in this way, because only then the symplectic condition \ref{symplectic_U_B} holds.  \par
This condition cannot be satisfied if not all eigenvalues are real. For example, if one eigenvalue is in $i\R$, the corresponding state is clearly not describable by ladder operators, i.e. the spectrum is no longer discretized, and is (dynamically) unstable. Then any small perturbation might irreversibly change the systems state from equilibrium \cite{PhysRevA.98.033610}. If at least one real eigenvalue is strictly negative, this indicates "a state [which] cannot be created adiabatically by gradually reducing the entropy associated with the thermal energy" \cite{PhysRevA.98.033610}. This is called a thermodynamic instability. Both types of instabilities are observed when the results of a Bogoliubov transformation are compared with the flow equations in Section \ref{Results}.
%%%%%%%%%%%%%%%%%%%%%%%%%%%%%%%%%%%%%%%%%%%%%%%%%%%%%%
\section{With Dependence on the Occupation Numbers}
\subsection{Useful Preliminaries}
Consider some operator $\hat f$ which depends on a single number operator $\hat n=\CR\AN$. The following relations will be used when deriving the flow equations:
\begin{subequations}
\label{fcom}
\begin{align}
\left[\CR,\hat f(\hat n) \right] &= \CR\left(\hat f(\hat n)-\hat f(\hat n+1)\right)\\ 
\left[\AN,\hat f(\hat n) \right] &= \AN\left(\hat f(\hat n)-\hat f(\hat n-1)\right)\\
\left[\hat f(\hat n),\CR \right] &= \left(\hat f(\hat n)-\hat f(\hat n-1)\right)\CR\\
\left[\hat f(\hat n),\AN \right] &= \left(\hat f(\hat n)-\hat f(\hat n+1)\right)\AN
\end{align}
\end{subequations}
These can be proved by induction for $\hat f(\hat n)=\hat n^k, k\in\N$ and from there simply extended to well-behaved $\hat f$ via power series. Equations \ref{fcom} are still valid for functions depending on $\left\{\hat n_k\right\}_k$, because all $\hat n_k$ pairwise commute.\\
We will write $\hat f\left(\hat n_1,\hat n_2,\hdots\right)\eqdef \hat f$ and $\hat f\left(\hat n_1,\hat n_2,\hdots,\hat n_k\pm 1,\hat n_{k+1},\hdots\right)\eqdef \hat f(\hat n_k\pm1)$. In this notation we define $\hat f(\hat n_k\pm1,\hat n_{k}\pm1)\defeq\hat f(\hat n_k\pm2)$.\\
A simple induction for $n_1,n_2\in\N_0$ yields the following useful relation:
\begin{align}
&\left[\hat f(\hat n),\CR_{k_1}\CR_{k_2}\cdots\CR_{k_{n_1}}\AN_{k_1}\AN_{k_2}\cdots\AN_{k_{n_2}} \right] \nonumber \\ \quad& 
= \left(\hat f-\hat f\left(\hat n_{k_1}-1,\hat n_{k_2}-1,\hdots,\hat n_{k_{n_1}},\hat n_{k_1}+1,\hat n_{k_2}+1\hdots\hat n_{k_{n_2}}+1\right)\right)\CR_{k_1}\CR_{k_2}\cdots\CR_{k_{n_1}}\AN_{k_1}\AN_{k_2}\cdots\AN_{k_{n_2}}
\end{align}
Furthermore, applying the recurrence relation \ref{recurrence relation} can be used to successively normal order operators. Let $\hat O\defeq \CR_{k_1}\CR_{k_2}\cdots\CR_{k_{n_1}}\AN_{k_1}\AN_{k_2}\cdots\AN_{k_{n_2}}$. Then normal ordering w.r.t. the vacuum yields:
\begin{subequations}
\begin{align}
\AN_q\NO\hat O\NO &=\NO\hat O\AN_q\NO + \sum\limits_{k}\NO \frac{\partial \hat O}{\partial\CR_k}\NO \nonumber \\
 &= \NO\hat O\AN_q\NO + \sum\limits_{i=1}^{n_1}\delta_{k_i,q}\NO \CR_{k_1}\CR_{k_2}\cdots\CR_{k_{i-1}}\CR_{k_{i+1}}\cdots\CR_{k_{n_1}}\AN_{k_1}\AN_{k_2}\cdots\AN_{k_{n_2}}\NO \\
\CR_q\NO\hat O\NO &= \NO \CR_q\hat O\NO
\end{align}
\end{subequations}
\subsection{Applying the Formalism}
Following the same same procedure as in the heavy impurity limit, we first start by evaluating the canonical generator. It turns out that $\hat\eta$ conserves the structure of the original Hamiltonian (cf. eq. \ref{psiphitheta_def}) while the flow Hamiltonian does not (cf. eq. \ref{com_eta_Hint_A} ff.) . Therefore, the sequence of higher and higher order terms has to be truncated at some point as discussed in Section \ref{Truncation Schemes}. 
Three simplifications will be made in order to obtain closed expressions for the flow equations:
\begin{itemize}
\item We will use a na\"ive and only partial normal ordering prescription where the contractions are defined with respect to the vacuum state and not the ground state of the diagonal Hamiltonian.
\item The considered expressions will not be fully normal ordered because the coefficients $\hat\omega_k $, $\hat W_{p,p^\prime}$, and $\hat V_{q,q^\prime}$ are not normal ordered. This saves the rather tedious process of normal ordering arbitrary functions of number operators, which involves expanding the operator into a Newton series \cite{10.21468/SciPostPhys.10.1.007}, but may render the sequence less well-behaved when truncated to an order as low as two.
\item  We will neglect all terms of order four or higher, i.e. terms which contain products of four creation/ annihilation operators or more.
\end{itemize}
After evaluating the commutator of $\hat\eta$ and the full Hamiltonian where we made frequent use of the equations in \ref{fcom}, we end up with the flow equations \ref{feq_ndep_I}-\ref{feq_ndep_IV}. In first order, we can expect the off-diagonal elements to vanish if $\hat H \neq \hat H(\hat n_q-1,\hat n_{q^\prime}+1)\ \forall q,q^\prime,q\neq q^\prime$ and $\hat H \neq \hat H(\hat n_p\pm 1,\hat n_{p^\prime}\pm 1)\ \forall p,p^\prime$ with \begin{equation}\hat H\defeq \Sum_k \hat\omega_k \NO\CR_k\AN_k\NO+\hat\epsilon. \end{equation}
The next section will address whether we can expect this to hold true.


\subsection{Discussion of the Applicability of the Flow Equations}
Discretizing $\ham_{\mathrm{LLP}}(p)$ can be done analogously to how it was done in the heavy impurity limit. We obtain:
\begin{align}
\ham_{\mathrm{LLP}}^{\mathrm{discr.}}(P) &=g_{IB}n_0+\frac{1}{2M}\left(p-\Sum_k k \CR_k\AN_k \right)^2+\Sum_k\omega_{ k}\CR_{ k}\AN_{ k}\nonumber\\
&+\sqrt{\frac{n_0\Delta k}{2\pi}}g_{IB}\Sum_k W_k \left(\AN_{ k}+\CR_{- k}\right) + \frac{g_{IB}\Delta k}{2\pi}\Sum_{k,k^\prime} \left(c_k\CR_k-s_k\AN_{-k}\right)\left(c_{k^\prime}\AN_{k^\prime}-s_{k^\prime}\CR_{-k^\prime}\right)
\end{align}
Because $\hat n_k=\CR_k\AN_k$ this can be written in the following way:
\begin{align}
\ham_{\mathrm{LLP}}^{\mathrm{discr.}}(p) &=\hat H(p)+ \frac{g_{IB}\Delta k}{2\pi}\Sum_{k\neq k^\prime}(c_kc_{k^\prime}+s_ks_{k^\prime})\CR_k\AN_{k^\prime}\nonumber\\
&+\sqrt{\frac{n_0\Delta k}{2\pi}}g_{IB}\Sum_k W_k \left(\AN_{ k}+\CR_{- k}\right) - \frac{g_{IB}\Delta k}{2\pi}\Sum_{k,k^\prime} \left(c_ks_{k^\prime}\CR_k\CR_{k^\prime}+s_kc_{k^\prime}\AN_{k}\AN_{k^\prime}\right)
\end{align}
$\hat H$ contains the parts of the Hamiltonian which can be written in terms of number operators:
\begin{equation}
\hat H(p)\defeq g_{IB}n_0+\frac{g_{IB}}{2\pi}\Delta k\Sum_{k}s_k^2+\frac{1}{2M}\left(p-\Sum_k k \hat n_k \right)^2+\Sum_k\omega_{ k}\hat n_k+ \frac{g_{IB}\Delta k}{2\pi}\Sum_{k}(c_k^2+s_k^2)\hat n_k
\end{equation}
Then for $q,q^\prime,q\neq q^\prime$ we get:
\begin{align}\label{n_test_eq_thinking}
&\hat H(p,\hat n_q-1,\hat n_{q^\prime}+1)-\hat H(p)=\frac{1}{2M}\left(p+q-q^\prime-\Sum_k k \hat n_k \right)^2-\frac{1}{2M}\left(p-\Sum_k k \hat n_k \right)^2\nonumber \\
&+\omega_{q^\prime}-\omega_q + \frac{g_{IB}\Delta k}{2\pi}(c_{q^\prime}^2+s_{q^\prime}^2-c_{q}^2-s_{q}^2) \\
&=\frac{q-q^\prime}{2M}\left(2p-2\Sum_k k \hat n_k+q-q^\prime\right)+\omega_{q^\prime}-\omega_q + \frac{g_{IB}\Delta k}{2\pi}(c_{q^\prime}^2+s_{q^\prime}^2-c_{q}^2-s_{q}^2) \label{dangerous_maths}
\end{align}
So in first order we can expect:
\begin{align}\label{nasty}
\hat V_{q,q^\prime}\sim \mathrm{exp}\left(-\left(\frac{q-q^\prime}{2M}\left(2p-2\Sum_k k \hat n_k+q-q^\prime\right)+\omega_{q^\prime}-\omega_q + \frac{g_{IB}\Delta k}{2\pi}(c_{q^\prime}^2+s_{q^\prime}^2-c_{q}^2-s_{q}^2)\right)^2\right)
\end{align}
Not worrying too much about the mathematical details like the fact that \ref{dangerous_maths} defines an unbounded operator because the number operators are unbounded, we can, loosely using the spectral mapping theorem \cite{Arendt1984} and the spectral radius formula, conclude that $\hat V_{q,q^\prime}$ should vanish if the spectrum of the argument of the exponential is a subset of $\R_{<0}$. Due to the square in the exponent the inclusion in $\R_{\leq 0}$ is clear.  \\
Considering the vacuum, 0 is in the spectrum of \ref{dangerous_maths} iff 
\begin{equation}
\frac{q-q^\prime}{2M}\left(2p+q-q^\prime\right)+\omega_{q^\prime}-\omega_q + \frac{g_{IB}\Delta k}{2\pi}(c_{q^\prime}^2+s_{q^\prime}^2-c_{q}^2-s_{q}^2)=0.
\end{equation}
This condition can hold for at most one $p$. For most other values of $p$, \ref{nasty} should be relatively well-behaved.
%Thanks to the term $\Sum_k k \hat n_k$ and because $q\neq q^\prime$ this operator is never the zero operator. 
%For fixed $q\neq q^\prime$ there will exist one and only one $p$ s.t. the expectation value of \ref{n_test_eq_thinking} is 0. For all other values of $p$, looking at the flow equations in first order, we expect the coefficients in front of $\CR_{q}\AN_{q^\prime}$ to vanish in the limit $\lambda\rightarrow\infty$.
The same reasoning applies to $\hat H(p,\hat n_p\pm1,\hat n_{p^\prime}\pm 1)-\hat H(p)$ and $\hat W_{p,p^\prime}$ to reach a similar conclusion.\\
This gives hope that flow equations \ref{feq_ndep_I}-\ref{feq_ndep_IV} converge to a diagonal Hamiltonian as desired for most values of $p$. Even if not, the second order terms of the flow equations may be enough to suppress the off-diagonal elements. So again, the convergence is best tested by numerically solving the flow equations, which will not be done in this thesis. \par
This involves the following steps: 
\begin{enumerate}
\item First, the displacement operator has to be applied to the full Hamiltonian. The condition that we want all linear terms to vanish will give a set of $N$ non-linear equations which are again to be solved for $\underline\alpha$. 
\item Then each coefficient appearing in the full Hamiltonian must be expanded in powers of $\hat n_k$. The resulting power series should not be truncated at less than quadratic order, otherwise nonlinearities will not be captured and the problem can be reduced to the case where none of the coefficients depend on the occupation numbers. \\ Even the coefficients which do not depend on the occupation numbers (such as $\omega_k$, $c_k$, $s_k$) must be expanded in terms of $\hat n_k$ because they can (and generally will) pick up non-trivial n-dependencies during the flow.
\item The flow equations \ref{feq_ndep_I}-\ref{feq_ndep_IV} (which define the flow for \emph{operators}) must be reduced to flow equations for the \emph{expansion coefficients} (see Appendix \ref{Systematically Expanding the Flow Equations}). A unpleasant but crucial point is that when the operators are expanded to order $n$, the number of evolution coefficients is $\mathcal O(N^n)$. But the number of operators is already $\mathcal O(N^2)$, so solving the resulting $\mathcal O(N^{2n})$ ODEs will be computationally expensive.
\item The resulting system of coupled ODEs can then be solved as in the heavy impurity limit.
\end{enumerate}
Instead of steps 2 and 3, an alternative and promising simplification of the flow equations might be to expand them in order $1/M$: The flow equations with $\hat n$-dependence contain the case without $\hat n$-dependence as a special case, and we know that these flow equations are exact for $1/M=0$. Dropping all terms of $\mathcal O(1/M^2)$ or higher may lead to flow equations that do not require any assumptions about or simplifications of the exact $\hat n$-dependence of the matrix elements to solve, as had to be made in step 2 above.





























































