\chapter{Introduction}
The flow equation approach was independently discovered by Franz Wegner \cite{https://doi.org/10.1002/andp.19945060203} and by Stanislaw D. Glazek and Kenneth G. Wilson \cite{PhysRevD.48.5863} as a new renormalization technique to diagonalize, or at least block-diagonalize, Hamiltonians. It is also known under the names Continuous Unitary Transformation (CUT), Double Bracket Flow, or Isospectral Flow, and has been successfully applied to a variety of physical systems, including the Kondo model, interacting bosons, and electron-phonon interaction \cite{Wegner_2006}.\par
This thesis deals with the application of the flow equation method to quadratic bosonic Hamiltonians. Two cases are distinguished: In the first case the Hamiltonian is purely quadratic, in the second case the Hamiltonian still has the same basic structure, but now the coefficients depend on (bosonic) occupation numbers.\\
The first case occurs, for example, in the BEC polaron problem in the limit of an infinite impurity mass. The dependence on occupation numbers comes into play when this impurity is not fixed, that is, when it has a finite mass.\par
%, or in the study of magnons in doped antiferromagnets.\par
This thesis is organized as follows: First, we will provide a brief introduction to the flow equation approach and elaborate on some of the problems it entails in Section \ref{Theoretical Background}. We will also give an overview on the Bose Polaron problem by discussing the Fröhlich Hamiltonian and the Lee-Low-Pine (LLP) Transformation. The flow equations  both for the purely quadratic case and the case where the coefficients depend on the bosonic occupation numbers will formulated in Section \ref{Determining the Flow Equations}. This section will be supplemented by detailed calculations in Appendix \ref{Detailed Calculations}\\
The flow equations for the purely quadratic case will be tested on a one dimensional Bose polaron model in Section \ref{Results} and compared to the findings of Grusdt \emph{et al.}\cite{Grusdt_2017}. \\
We then close with a conclusion and an outlook in Section \ref{Conclusion and Outlook}.
