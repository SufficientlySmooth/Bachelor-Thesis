\chapter{Introduction}
The flow equation approach was independently discovered by Franz Wegner \cite{https://doi.org/10.1002/andp.19945060203} and by Stanislaw D. Glazek and Kenneth G. Wilson \cite{PhysRevD.48.5863} as a new renormalization technique to diagonalize, or at least block-diagonalize, Hamiltonians. It is also known under the names Continuous Unitary Transformation (CUT), Double Bracket Flow, or Isospectral Flow, and has been successfully applied to a variety of physical systems, including the Kondo model, interacting bosons, and electron-phonon interaction \cite{Wegner_2006}. The fact that sometimes it can still be applied in cases where conventional techniques such as perturbation theory fail makes the flow equation approach an especially powerful and interesting tool for many-body problems.  It consists of a continuous unitary transformation of the Hamiltonian, guided by a generator that makes the Hamiltonian increasingly diagonal. \par
This thesis deals with the application of the flow equation method to quadratic bosonic Hamiltonians. Two cases are distinguished: In the first case the Hamiltonian is purely quadratic, in the second case the Hamiltonian still has the same basic structure, but now the matrix elements of the second quantized Hamiltonian depend on the (bosonic) occupation numbers.\\
The latter occurs, for example, in the Bose polaron problem with a movable impurity of finite mass. The dependence on the occupation numbers vanishes in the limit of an infinite impurity mass, that is, when the impurity is fixed.\par
%, or in the study of magnons in doped antiferromagnets.\par
The structure of this thesis is as follows: First, we will provide a brief introduction to the flow equation approach and elaborate on some of the problems it entails in Section \ref{Theoretical Background}. We will also give an overview on the Bose Polaron problem by discussing the Fröhlich Hamiltonian and the Lee-Low-Pine (LLP) Transformation. The flow equations  both for the purely quadratic case and the case where the coefficients depend on the bosonic occupation numbers will formulated in Section \ref{Determining the Flow Equations}. This section will be supplemented by detailed calculations in Appendix \ref{Detailed Calculations}.\\
In Section \ref{Results} the flow equations  for the purely quadratic case will be tested on the dimensional Bose polaron model by comparing its result with the spectrum that can be obtained by an exact diagonalization of the corresponding Hamiltonian with the help of Bogoliubov transformations.\\
%The flow equations for the purely quadratic case will be tested on a one dimensional Bose polaron model in Section \ref{Results} and compared to the findings of Grusdt \emph{et al.}\cite{Grusdt_2017}. \\
We then close by discussing the advantages and limitations of our method, as well as possible extensions and future applications in Section \ref{Conclusion and Outlook}.% with a conclusion and an outlook 
