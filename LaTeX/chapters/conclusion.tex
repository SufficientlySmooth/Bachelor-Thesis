\chapter{Conclusion and Outlook}\label{Conclusion and Outlook}
The overarching goal of this project was to test whether the flow equation approach can be successfully applied to quadratic bosonic Hamiltonians. We can confirm this for the purely quadratic case where none of the coefficients of the second quantized Hamiltonian depend on the occupation numbers, since we were able to reproduce the spectrum as is obtained by exact diagonalization for the 1D Bose Polaron Problem in the heavy impurity limit. We showed that convergence to the spectrum can only be achieved by introducing twisted boundary conditions which (ever so slightly) break the symmetry of the Bogoliubov dispersion. This should not be considered a particularly large limitation of the method, since an arbitrarily small perturbation is sufficient to break the symmetry. \par
A far more significant drawback of flow equations is that solving them has turned out to be computationally expensive. $\mathcal O(N^2)$ coupled ODEs have to be solved which is far more intricate that solving the eigenvalue problem for a matrix of size $2N$ when performing a Bogoliubov Transformation. It is therefore explicitly not recommended to use the flow equations to diagonalize a purely quadratic bosonic Hamiltonian. At the same time, however, it should be noted that the eigenvalue problems to be solved for the Bogoliubov transform have been numerically optimized for decades, while the code written for this thesis is not necessarily particularly efficient or well optimized. The biggest attempt to improve performance made here was done by switching the programming language from Python to Julia.\par
Due to the computational complexity of the flow equation approach, it is not reasonable to apply it to a significantly finer grid, or to a grid with increased UV cutoff or decreased IR cutoff while maintaining the same spacing $\Delta k$. A grid with adaptive spacing might be a good way to overcome this issue. By introducing logarithmic spacing between the $k-$values it would be possible to cover a large part of momentum space with reasonable computational effort. The changes that would need to be made to the existing code to implement such a grid and solving the flow equations on that grid would be minimal.   \par
Another good approach to improve the performance of the flow equations would be to adaptively stop the flow when convergence is reached by implementing an appropriate termination condition for the numerical integrator, i.e. to check if the change of a matrix element during the last integration steps has fallen below a certain threshold. It is even conceivable to use different termination points for matrix elements. As discussed in the last section, this would mean that high energy parts of the Hilbert space would be terminated first, because they settle relatively quickly to a constant value for the matrix elements and the spectrum. Of course, care must be taken that the overhead for the integrator to check for a termination condition is not greater than the computational savings from reducing the effective number of ODEs. If implemented well, this promises to reduce the computational complexity of solving the flow equations very substantially. However, this would require equally substantial changes to the code developed for this project. \par
To summarize, we are particularly pleased that in this thesis a very good agreement between the ground state energy via exact diagonalization and the flow equations was found. This gives hope that even after introducing some approximations the flow equations are still close to the exact solution for other more complex problems.\par
So the prospect of applying the flow equation approach to other problems where coefficients of the second quantized Hamiltonian depend on the occupation numbers is promising. In this case, exact diagonalization is no longer possible, whereas the generic flow equations derived in this thesis are still applicable. Whether they exhibit the desired convergence behavior must be tested for each specific application and may involve intricacies such as the necessity to introduce twisted boundary conditions. 
Those possible future applications may include the 1D Bose polaron problem at finite impurity mass or the description of magnons in \mbox{antiferromagnets \cite{Bermes_2023}}.
